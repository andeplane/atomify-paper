\documentclass[aps,pre,twocolumn,letterpaper,floatfix,showpacs]{revtex4}
\usepackage{graphicx} 
\usepackage{amsmath,amssymb,amsfonts} 
\usepackage{mathtools}
\usepackage{pdfpages}
\usepackage{afterpage}
\usepackage[hidelinks]{hyperref} 
\usepackage{epstopdf}
\begin{document}
\title{Atomify - a live LAMMPS visualizer}
\author{Anders Hafreager$^1$}
\author{Svenn-Arne Dragly$^{1}$} 
\author{Anders Malthe-S\o renssen$^1$}
\affiliation{$^1$Department of Physics - University of Oslo\\Sem S{\ae}lands vei 24, NO-0316, Oslo, Norway }
\date{\today} 

%%%%%%%%%%%%%%%%%%%%%%%%%%%%%%%%%%%%%%%%%%%%%%%%%%%%%%%%%%%%%%
%%%%%%%%%%%%%%%%%%%%%%%%%%%%%%%%%%%%%%%%%%%%%%%%%%%%%%%%%%%%%%
\begin{abstract} 
The typical workflow when developing LAMMPS scripts includes working with several programs. A text editor is needed to modify the scripts, the terminal to run the simulation, and programs like VMD or Ovito to visualize the system over time. If physical quantities are computed, the data is often plotted with MATLAB or Python. This is a tedious process, especially for teaching purposes and for people who are new to LAMMPS. 
We here introduce Atomify; a high performance live visualizer for LAMMPS simulations, with stunning graphics able to simulate and render more than 250000 atoms with excellent frame rate on modern hardware. Atomify supports OpenMP acceleration, live plotting of LAMMPS variables and computes, and an easy-to-use code editor in one single program. Direct access to the powerful machinery already built into LAMMPS allows easy access to advanced physical quantities. Atomify is open source software (GPL) written in C++, built on top of Qt. 
\end{abstract} 
 
\maketitle
 
% %%%%%%%%%%%%%%%%%%%%%%%%%%%%%%%%%%%%%%%%%%%%%%%%%%%%%%%%%%%%%
%% %%%%%%%%%%%%%%%%%%%%%%%%%%%%%%%%%%%%%%%%%%%%%%%%%%%%%%%%%%%%
\section{Introduction}
Molecular dynamics has become a standard technique for simulating atoms and molecules in a broad range of fields. Over the years, increasingly sophisticated methods have been developed. With the large number of force fields, advanced techniques and GPU acceleration, implementing a good molecular dynamics program from scratch a challenging task. While many people at some point write their own code, most end up using one of the many available codes\footnote{See \url{https://en.wikipedia.org/wiki/Comparison_of_software_for_molecular_mechanics_modeling}} such as LAMMPS\cite{plimpton1995fast} and GROMACS\cite{Pronk2013}. Both of these codes has been developed for more than two decades and contains more than one million lines of code each. 

When 


\section{Features}


\section{Implementation}
Atomify is a rather small code built on Qt. All GUI is defined in QML whereas the backend of the program communicating with LAMMPS is written in c++. Simply put, Atomify runs an instance of a LAMMPS object enabling the user to visualize the current state of LAMMPS and interact with LAMMPS while the simulation is running. 

\subsection{Communicating with LAMMPS}

\subsection{Live plotting}

\subsection{Data pipeline}
When rendering one often wants to modify the data from LAMMPS before it is visualized. Examples are periodic images of the simulation, slicing and coloring. We have been inspired by the pipeline used in Ovito where the data flows through several \textit{modifiers}, each modifying the visualization data before the data is converted to a format for the GPU.

\subsection{Rendering techniques}
Simulations of atoms are often visualized using spheres and cylinders. The spheres represent the atoms while the cylinders represent the bonds between the atoms. This way the molecular structure is easy to understand. In many visualization tools, the spheres and cylinders are built up by many triangles connected so they form the geometrical object of interest. The problem with this rendering technique is that each sphere needs many triangles in order to look spherical. This significantly reduces the rendering performance. A common technique to overcome this problem is to use billboards. Billboards are planes made up by two triangles with surface normal parallel to the vector pointing towards the camera. By doing proper shading, the billboards can mimick the spheres and cylinders so they look realistic.

\subsubsection*{Sphere billboards}
Each sphere is visualized as two triangles making up a square large enough to cover all pixels the sphere needs. 

\subsubsection*{Cylinder billboard}

\section{Case study}


\section{Conclusion}
We have

\subsection{Future work}
All parts of Atomify can easily be ported to the web using WebGL and Emscripten. 

\bibliographystyle{plain}
\bibliography{Remote}

\end{document}  